\section{Database}
\label{sect:database}
As ProGENitor needed a method to pull large amounts of data off the back end
server database, a method had to be implemented that interfaced with
the database.  MySQL was chosen as it is open source, widely used, and fairly
easy to quickly learn.  Once the MySQL interface was established a wrapper was
added such that another database method could be inserted without a significant
work effort in the rest of the code.  Then, through this wrapper, the many
different function calls were implemented to collect the data needed to generate
the career path graph and derive any further insights through Weka.

\subsection{Database Interface}
To greatly simplify this work, a predefined library was added to the project.  The 
library allows for easy access to a database for creating, reading from,
writing to, and querying the database.  In the case of ProGENitor, once the
library was added, the code was very straight forward.  Through a couple
commands, the code established the database connection, ran the specified query
or other command, and collected the returned data \cite{sqlite}.  Using the
library allowed the interface to quickly add in functions to create
a database, collect query matches from the database, upload lines and files to
the database, modify lines within the database, and even pull the entire
database.  With these functions in place, ProGENitor easily and quickly can
access any defined database.  In the case of ProGENitor, four databases were
generated; one for user profile data, one for education data, one for job data,
and finally one that contains the headers of the other databases.  If, in the
future, additional databases are needed, ProGENitor can easily add them. 
Additionally, as the SQL commands are standard commands, the interface can
easily be replaced with another database interface or expanded upon by anyone
familiar with a SQL language.

\subsection{User Wrapper}
As previously stated, it was desired that ProGENitor be setup such that it was
easy to swap out the database interface with another interface.  Thus, the user
interface wrapper was written to call the various SQL commands.  If in the
future, the database needed to be changed, the work to do so would reside in
adding the database interface and changing the user interface wrapper to point
to the new database.  The wrapper also adds in commands that make interfacing
with the code a bit more clear.  Commands such as add user, database
setup, query matching users, and return headers all allow users working through
other portions of code to understand what the function calls are actually doing
and do not require the developer to necessarily understand the database
interface commands.  Then using these commands, ProGENitor can collect data that
it passes along to be processed by the career path graphing and Weka packages.

\subsection{Data Collection}
With the wrapper and SQL interface in place, ProGENitor then implements a couple
different calls to gather data to be analyzed.  The first of these calls code
that polls the data base for all users that match the query field value passed
to the method.  This method then returns the user IDs in a set for all of the
users that matched the query.  The next data collection method available does
the same function, but instead returns all of the matching data in a list.  The
final data collection method available returns a list containing all of the
data associated with the set of IDs passed to the method.  These methods are all
very similar, but allow for easy data collection by the career path graphing and
Weka methods.

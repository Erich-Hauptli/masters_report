\section{Database}
\label{sect:database}
As ProGENitor needed a method to pull large amounts of data off the backend
server database, a Java method needed had to be implemented that inferfaced with
the database.  MySQL was chosen as it is open source, widely used, and fairly
easy to learn quickly.  Once the MySQL interface was established a wrapper was
added such that another database method could be inserted without a significant
work effort in the rest of the code.  Then, through this wrapper, the many
different function calls wer implemented to collect the data needed to generate
the career map and derive any further insights through Weka.

\subsection{MySQL Interface}
To greatly simplify this work, a predefined library jar was added
to to the project.  In this case, the library used was SQLite JDBC.  The SQLite
library allows for easy access to a MySQL database for creating, reading from,
writting to, and querrying the database.  In the case of ProGENitor, once the
library was added, the code was very straight forward.  Through a couple
commands, the code established the database connection, ran the specified query
or other command, and collected the returned data.\cite{sqlite}  Using the
SQLite library allowed the MySQL interface to quickly add in functions to create
a database, collect query matches from the database, upload lines and files to
the database, modify lines within the database, and even pull the entire
databse.  With these functions in place, ProGENitor quick easily and quickly
access any defined database.  Additionaly, as the SQL commands are standard
commands, the interface can easily be be replaced with another database
interface or expanded upon by anyone familiar with an SQL language.

\subsection{User Wrapper}


\subsection{Data Collection}


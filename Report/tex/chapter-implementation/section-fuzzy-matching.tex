\section{Fuzzy Matching}
\label{sect:fuzzy-matching}
One of the challenges with processing the data of a database such as LinkedIN is
that the data is free form.  Although ProGENitor makes no attempt of matching
similar jobs or other data points, it does attempt to account for minor spelling
differences.  Thus, if the users misspell a word or use a slightly different
spelling, the similarities will still be captured. This fuzzy matching is
done by using the Levenshtein distance algorithm\cite{fuzzy} outlined in table
\ref{tab:lev-dist}.  Once the algorithm calculates the difference between two
words, it then checks to see if the difference is within the acceptable range. 
Currently this range is set to less than or equal to two.  If the difference is
acceptable, ProGENitor will consider the two words identical for matching
purposes.


\begin{table}[H]
  \caption{Levenshtein Distance Algorithm}
  \centering
  \begin{tabular}{|p{.5in}|p{4in}|}
  \hline
  \
  %heading
  Step & Description \\
  \hline\hline
  1 &  Set n to be the length of s.\newline 
  Set m to be the length of t.\newline
  If n = 0, return m and exit.\newline
  If m = 0, return n and exit.\newline
  Construct a matrix containing 0..m rows and 0..n columns.  \\ \hline 
  2 &  	Initialize the first row to 0..n.\newline
  Initialize the first column to 0..m.\\ \hline 
  3 & Examine each character of s (i from 1 to n). \\ \hline
  4 & Examine each character of t (j from 1 to m). \\ \hline
  5 &  	If s[i] equals t[j], the cost is 0.\newline
  If s[i] doesn't equal t[j], the cost is 1. \\\hline 
  6 &  	Set cell d[i,j] of the matrix equal to the minimum of:\newline
  a. The cell immediately above plus 1: d[i-1,j] + 1.\newline
  b. The cell immediately to the left plus 1: d[i,j-1] + 1.\newline
  c. The cell diagonally above and to the left plus the cost: d[i-1,j-1] + cost.\\ \hline
  7 & After the iteration steps (3, 4, 5, 6) are complete, the distance is found in cell d[n,m]. \\ \hline
  \end{tabular}
  \label{tab:lev-dist}
\end{table}


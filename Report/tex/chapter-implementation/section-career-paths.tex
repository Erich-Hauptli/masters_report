\section{Career Paths}
\label{sect:career-paths}
Section Description.  

\subsection{Node Interconnects}
Node Interconnect Generation:
\begin{enumerate}
  \item For each ID passed to Interconnect Generation Module:
  \begin{enumerate}
    \item Pull Job Data and add it to the Nodes Array List.
    \item Pull Education Data and add it to the Nodes Array List.
    \item Set Min equal to MAX Integer and Max equal to MIN Integer.
    \item For each element of the Nodes ArrayList:
    \begin{enumerate}
      \item If date of data for element of Nodes is less than Min and more than
      Max, store the data and set Min equal to date of data.
      \item After all elements of Nodes ArrayList considered, Add stored data
      to User ArrayList and set Max equal to Min.
  	\end{enumerate}
  	\item For each element of User ArrayList:
  	\begin{enumerate}
  	  \item If A is NULL, set A equal to User element node name.
  	  \item Else set B equal to A and set A equal to User element node name.
  	  \item If Connects ArrayList is empty, add B,A,1 to Connects ArrayList.
  	  \item Else check if B,A exists in the Connects ArrayList:
  	  \begin{enumerate}
  	    \item If it exists, increment the counter of the row.
  	    \item If it does not exist, add B,A,1 to the Connects ArrayList.
  	  \end{enumerate}
  	\end{enumerate}
  \end{enumerate}
  \item Push Connects ArrayList containing all node transitions and transition
  counts to a JSON Object containing a JSON Array.
  \item Return both the ArrayList and the JSON Object.
\end{enumerate}

\subsection{Node Ordering}
Node Ordering Generation:
 \begin{enumerate}
   \item Generate HashSet of All Nodes
   \item For each Node in HashSet:
   \begin{enumerate}
     \item Initialize transitional weight to 0.
     \item For each element of the Node Interconnect Array List:
     \begin{enumerate}
       \item Check if Node matches the input node.
       \item Check if the number of transitions to the node is greater than the
       transitional weight.
       \item If both checks are true, set the transitional weight to the current
       Array List line's number of transitions.
       \item Also, if both checks are true, store this Array List line.
     \end{enumerate}
     \item After the worst input transition is found for the node, store it in
     the Heavy Edges HashSet.
     \item Repeat this entire step for the output nodes.
   \end{enumerate}
   \item For each Heavy Edge element, search the Node Interconnect Array List
   for input nodes that are also destination nodes.
   \begin{enumerate}
     \item Any nodes not found are set as start nodes.
     \item Repeat this step for output nodes that are also starting nodes.  Any
     nodes not found are set as ending nodes.
   \end{enumerate}
   \item Add all the starting nodes to node 0 and add them to the Node Store
   HashSet.
   \item Add all nodes that are not starting nodes to the Remaining Nodes
   HashSet.
   \item Increment the group number to 1.
   \item Until the Remaining Nodes HashSet is empty, loop through the following
   steps.
   \begin{enumerate}
     \item For each Node in Node Store, store all destination nodes in a HashSet
     that Node in Node Store transitions to.
     \item For each destination node stored in the previous step, find all
     possible next destination nodes and check if they are contained within the
     HashSet generated in the previous step.
     \begin{enumerate}
       \item If one is contained within the previously generated HashSet, remove
       the node from the HashSet.
     \end{enumerate}
     \item Add remaining nodes to next node grouping.  Also remove remaining
     nodes from Remaining Nodes HashSet.
     \item Add the node group to the NodeReturn ArrayList.
     \item Increment the group number.
     \item Replace the nodes in the Starting Nodes hash set with the nodes that
     were just added to a group.
   \end{enumerate} 
   \item Generate a JSON Object containing a JSON Array the node groupings from
   the NodeReturn ArrayList.
   \item Return both the JSON Object and the NodeReturn ArrayList.
 \end{enumerate}



\subsection{Node Details}

Node Detail Generation:
\begin{enumerate}
  \item Pull in Profile ArrayList, tag each element as a profile, and then add
  the element to the Complete ArrayList.
  \item Repeat this for the Jobs ArrayList and the Education ArrayList.
  \item Check each element fo the Complete ArrayList.
  \begin{enumerate}
    \item If the element contains the node that details are being pulled on, add
    the element to the Relevant ArrayList.
  \end{enumerate}
  \item Pull the headers associated with the node that details are being pulled
  on.
  \item Pull all the data in the database for that node and store in the All
  Node Data ArrayList.
  \item For each element of the Complete ArrayList:
  \begin{enumerate}
    \item Split the element into columns and step through each column.
    \begin{enumerate}
    	\item Check if the column element is a start or end year and instead
    	calculate the years spent at the node.
    	\begin{enumerate}
    	  \item If the end year is set to current, find the current year and then
    	  calculate the total years spent at the node.
    	\end{enumerate}
    	\item Add the column value to a HashSet to obtain all possible values for
    	the column.
    	\item Step through the colum counting each value instance to obtain a count
    	for each different value.
    	\item Calculate the percentage for each value in the column by diving the
    	count by the total number of elements.
    	\item Push these values into the Relevant ArrayList.
    \end{enumerate} 
  \end{enumerate}
  \item Repeat for each element of the All Node Data Array List
  \item Compare the percentages for each element of the Relevant ArrayList to
  the percentages from the All Node Data Array List.
  \begin{enumerate}
    \item Flag the column value for any instance where the Relevant value's
    percentage exceeds the percentage for all the data by 5\%.
    \item Return this value as relevant so that it can be identified to the user
    as significant to the node.
  \end{enumerate}
  \item Return the Relevant ArrayList and a JSON Object containing a JSON Array
  of the same data.
\end{enumerate}


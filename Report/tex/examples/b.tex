\section{design b}
\label{sect:design-b}
design b description.   

\subsection{A}
Subsection A Description.  

\section{Modifications to Programmable MBIST}
\label{sect:bg-modifications}
Text

\subsection{Address Generator Expansion}
More Text and 2\textsuperscript{\textit{i}} CMs output.  Table \ref{tab:ac}
provides an example of the address complement design and Table \ref{tab:gc} shows the same for the Gray code design.  In these tables, \textit{A\textsubscript{3:0}} is the next address and \textit{C\textsubscript{3:0}} is the current output of the linear counter.

\begin{table}[H]
  \centering
  \caption[Address Complement CM Algorithm]{Address Complement Algorithm for 4-bit Address Lines}
  \begin{tabular}{|l|l|}
    \hline
    \multicolumn{2}{|c|}{Address Complement} \\
    \hline  
    Up  & Down  \\                                                          
    \hline
    A\textsubscript{3} = C\textsubscript{0}                           & A\textsubscript{3} = $\overline{C\textsubscript{0}}$              \\ %ac3
    A\textsubscript{2} = C\textsubscript{3} $\oplus$ C\textsubscript{0} & A\textsubscript{2} = C\textsubscript{3} $\oplus$ C\textsubscript{0} \\ %ac2
    A\textsubscript{1} = C\textsubscript{2} $\oplus$ C\textsubscript{0} & A\textsubscript{1} = C\textsubscript{2} $\oplus$ C\textsubscript{0} \\ %ac1
    A\textsubscript{0} = C\textsubscript{1} $\oplus$ C\textsubscript{0} & A\textsubscript{0} = C\textsubscript{1} $\oplus$ C\textsubscript{0} \\ %ac0
    \hline
  \end{tabular} 
  \label{tab:ac}
\end{table}

\begin{table}[H]
  \centering
  \caption[Gray Code CM Algorithm]{Gray Code Algorithm for 4-bit Address Lines}
  \begin{tabular}{|l|l|}
    \hline
    \multicolumn{2}{|c|}{Gray Coding} \\
    \hline  
    Up & Down \\
    \hline
    A\textsubscript{3} = C\textsubscript{3}                           & A\textsubscript{3} = $\overline{C\textsubscript{3}}$              \\ %gc3
    A\textsubscript{2} = C\textsubscript{3} $\oplus$ C\textsubscript{2} & A\textsubscript{2} = C\textsubscript{3} $\oplus$ C\textsubscript{2} \\ %gc2
    A\textsubscript{1} = C\textsubscript{2} $\oplus$ C\textsubscript{1} & A\textsubscript{1} = C\textsubscript{2} $\oplus$ C\textsubscript{1} \\ %gc1
    A\textsubscript{0} = C\textsubscript{1} $\oplus$ C\textsubscript{0} & A\textsubscript{0} = C\textsubscript{1} $\oplus$ C\textsubscript{0} \\ %gc0
    \hline  
  \end{tabular}
  \label{tab:gc}
\end{table} 

The 2\textsuperscript{\textit{i}} CM. For \textit{N} lines, the regular
2\textsuperscript{\textit{i}} requires \textit{2}x\textit{N}x\textit{N}=\textit{2N}\textsuperscript{i} mux inputs.  This number can be reduced to \textit{2N}+(\textit{2}x\textit{2}x\textit{N})=\textit{6N} mux inputs using the minimal 2\textsuperscript{\textit{i}} CM \cite{5941430}.  Table \ref{tab:2i} provides an example of the regular and minimal solution for a 4-bit address line.  In the regular solution, the address columns are rotating to the left as \textit{i} increases.  This requires a 4-input mux on each of the 4 address lines plus an additional input required for up/down address direction.  The minimal solution recognizes the C\textsubscript{0} input toggels on each count and interchanges the C\textsubscript{\textit{i}} column with the C{\textsubscript{0}} column.

\begin{table}[H]
  \caption{2\textsuperscript{\textit{i}} Addressing Example}
  \centering
  \begin{tabular}{|c||c|c|c|c||c|c|c|c|}
  \hline
    \multicolumn{1}{|c||}{} & 
    \multicolumn{4}{|c||}{Regular 2\textsuperscript{\textit{i}}} &
    \multicolumn{4}{|c|}{Minimal 2\textsuperscript{\textit{i}}} \\
  \hline
   \textit{num.}&  0 &             1 &             2 &             3 &             0 &             1 &             2 &             3 \\
   \hline
   0 & 000\textbf{0} & 00\textbf{0}0 & 0\textbf{0}00 & \textbf{0}000 & 000\textbf{0} & 00\textbf{0}0 & 0\textbf{0}00 & \textbf{0}000 \\  
   1 & 000\textbf{1} & 00\textbf{1}0 & 0\textbf{1}00 & \textbf{1}000 & 000\textbf{1} & 00\textbf{1}0 & 0\textbf{1}00 & \textbf{1}000 \\  
   2 & 001\textbf{0} & 01\textbf{0}0 & 1\textbf{0}00 & \textbf{0}001 & 001\textbf{0} & 00\textbf{0}1 & 0\textbf{0}10 & \textbf{0}010 \\  
   3 & 001\textbf{1} & 01\textbf{1}0 & 1\textbf{1}00 & \textbf{1}001 & 001\textbf{1} & 00\textbf{1}1 & 0\textbf{1}10 & \textbf{1}010 \\  
   \hline                                                                          
   4 & 010\textbf{0} & 10\textbf{0}0 & 0\textbf{0}01 & \textbf{0}010 & 010\textbf{0} & 01\textbf{0}0 & 0\textbf{0}01 & \textbf{0}100 \\  
   5 & 010\textbf{1} & 10\textbf{1}0 & 0\textbf{1}01 & \textbf{1}010 & 010\textbf{1} & 01\textbf{1}0 & 0\textbf{1}01 & \textbf{1}100 \\  
   6 & 011\textbf{0} & 11\textbf{0}0 & 1\textbf{0}01 & \textbf{0}011 & 011\textbf{0} & 01\textbf{0}1 & 0\textbf{0}11 & \textbf{0}110 \\  
   7 & 011\textbf{1} & 11\textbf{1}0 & 1\textbf{1}01 & \textbf{1}011 & 011\textbf{1} & 01\textbf{1}1 & 0\textbf{1}11 & \textbf{1}110 \\  
   \hline                                                                          
   8 & 100\textbf{0} & 00\textbf{0}1 & 0\textbf{0}10 & \textbf{0}100 & 100\textbf{0} & 10\textbf{0}0 & 1\textbf{0}00 & \textbf{0}001 \\  
   9 & 100\textbf{1} & 00\textbf{1}1 & 0\textbf{1}10 & \textbf{1}100 & 100\textbf{1} & 10\textbf{1}0 & 1\textbf{1}00 & \textbf{1}001 \\  
  10 & 101\textbf{0} & 01\textbf{0}1 & 1\textbf{0}10 & \textbf{0}101 & 101\textbf{0} & 10\textbf{0}1 & 1\textbf{0}10 & \textbf{0}011 \\  
  11 & 101\textbf{1} & 01\textbf{1}1 & 1\textbf{1}10 & \textbf{1}101 & 101\textbf{1} & 10\textbf{1}1 & 1\textbf{1}10 & \textbf{1}011 \\  
  \hline                                                                           
  12 & 110\textbf{0} & 10\textbf{0}1 & 0\textbf{0}11 & \textbf{0}110 & 110\textbf{0} & 11\textbf{0}0 & 1\textbf{0}01 & \textbf{0}101 \\  
  13 & 110\textbf{1} & 10\textbf{1}1 & 0\textbf{1}11 & \textbf{1}110 & 110\textbf{1} & 11\textbf{1}0 & 1\textbf{1}01 & \textbf{1}101 \\  
  14 & 111\textbf{0} & 11\textbf{0}1 & 1\textbf{0}11 & \textbf{0}111 & 111\textbf{0} & 11\textbf{0}1 & 1\textbf{0}11 & \textbf{0}111 \\  
  15 & 111\textbf{1} & 11\textbf{1}1 & 1\textbf{1}11 & \textbf{1}111 & 111\textbf{1} & 11\textbf{1}1 & 1\textbf{1}11 & \textbf{1}111 \\  
  \hline
  \end{tabular}
  \label{tab:2i}
\end{table}

\section{Area Comparisons}
requires 0.6\% more area.

\label{sect:cln-area}
\subsection{Address Generator Area}
Table \ref{table:ac_area_compare} below shows the area increases by  59.3\% for the address counter block and 17.7\% for the overall PMBIST area.  

\begin{table}[H]
\caption[Address Counter Area Comparison]{Address Counter Area Comparison(\textit{um\textsuperscript{2}})}
\centering
\begin{tabular}{| l | l | l | l |}
\hline
Component & Base Design & Proposed Design & Percentage Difference \\ [0.5ex]
\hline\hline
address counter & 1810   & 2884   & 59.3\% \\
pmbist area     & 6057   & 7132   & 17.7\% \\ 
\hline
\end{tabular}
\label{table:ac_area_compare}
\end{table}

It is also important to note Table \ref{table:ac_area_overhead} shows considered.   

\begin{table}[H]
\caption{Address Counter Area within Memory Block}
\centering
\begin{tabular}{p{0.5in} p{1.25in} | l | l | l |  }
\cline{3-5}
& & \multicolumn{3}{ c| }{Area Overhead Percentages} \\
\hline
\multicolumn{1}{|p{0.5in}|}{Memory Size} & Total Memory Area (\textit{um\textsuperscript{2}}) & Base Design & Proposed Design & Difference \\ [1ex]
\hline\hline
\multicolumn{1}{|c|}{64x8  }  & 15780  & 27.7\% & 31.1\% & 3.4\% \\
\multicolumn{1}{|c|}{128x8 }  & 16884  & 26.4\% & 29.7\% & 3.3\% \\
\multicolumn{1}{|c|}{256x8 }  & 19333  & 23.9\% & 26.9\% & 3.1\% \\
\multicolumn{1}{|c|}{512x8 }  & 24108  & 20.1\% & 22.8\% & 2.7\% \\
\multicolumn{1}{|c|}{1024x8}  & 34013  & 15.1\% & 17.3\% & 2.2\% \\
\multicolumn{1}{|c|}{2048x8}  & 53276  & 10.2\% & 11.8\% & 1.6\% \\ 
\multicolumn{1}{|c|}{4096x8}  & 92093  & 6.2\%  & 7.2\%  & 1.0\% \\
\multicolumn{1}{|c|}{8192x8}  & 172577 & 3.4\%  & 4.0\%  & 0.6\% \\ [1ex]
\hline
\end{tabular}
\label{table:ac_area_overhead}
\end{table}


\subsection{Pattern Generator Area}
Table \ref{tab:pg_memory_compare} auxiliary memory.

\begin{table}[H]
\caption{Area of Pattern Generator Compared to Auxiliary Memory}
\centering
\begin{tabular}{|c| c| c|}
\hline
Memory Size & Memory Area & Area Reduction \\ [0.5ex]
\hline\hline
4x8   & 10289 & 78.6\%  \\
8x8   & 11393 & 80.6\%  \\
16x8  & 12497 & 82.4\%  \\
32x8  & 13601 & 83.8\%  \\
64x8  & 14706 & 85.0\%  \\
128x8 & 15809 & 86.0\%  \\
256x8 & 18258 & 87.9\%  \\
512x8 & 23033 & 90.4\%  \\
\hline
\end{tabular}
\label{tab:pg_memory_compare}
\end{table}

\subsubsection{C}
Subsection C Description.  



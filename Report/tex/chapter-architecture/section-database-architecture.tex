
\section{Database Architecture}
\begin{figure}[H]
	\setlength{\unitlength}{0.14in} % selecting unit length
	\centering % used for centering Figure
	\begin{picture}(35,12) % picture environment with the size (dimensions)
		% 32 length units wide, and 15 units high.
		\put(0,0){\framebox(20,12){}}
		\put(5,2){\framebox(10,7){MySQL Interface}}
		\put(25,1){\framebox(10,10){Data Collection}}
		\put(20,6){\vector(1,0){5}}
		\put(6.5,10) {User Interface}
	\end{picture}
	\caption{Database Block Diagram} % title of the Figure
	\label{fig:dbblock} % label to refer figure in text
\end{figure}
As depicted above in figure \ref{fig:dbblock}, the
ProGENitor database code is designed around a MySQL database. The code is
modular, so that any alternate database type can be inserted to allow for a
quick transition to another database architecture.  Currently the database is
broken into four different databases, but more can easily be added if necessary.
These databases contain information on each individual user, user job history,
user education history, and a listing of all of the headers for each database. 
The database SQL code allows for reading from, writing to, and creating the databases.  The
user wrapper creates a single point that the SQL interface would need to be
modify to support a change in database architecture.  Additionally, it
provides some basic query commands to pull data from the database based on a
search field.  It can also extract Meta data about the databases.  The database
is this pulled through a data collection package.  The data collection package
pulls data based on either a similar query field, a similar node, or a similar
user id.


\subsection{Database Technology Stack}
For the database in ProGENitor, a MySQL database was used.  This decision was
made not because it was the optimum solution for the data set, but because MySQL
is easier to learn and manage.  As the point of the project was not focusing on
databases, but on the analytics side of what can be done with the data, the goal
was to get a database setup quickly with less effort.  NoSQL databases are
better for unstructured data and in many instances, databases containing career
information, such as LinkedIN have chosen to go the NoSQL route.  For instance,
LinkedIN uses a database called Sensei\cite{sensei}, which is a NoSQL based
database.  Many NoSQL databases are proprietary where as the MySQL database
follows a standard that the entire database community is familiar with.  Thus,
ProGENitor was built around a MySQL database, as many of the interface commands
are similar for both NoSQL and MySQL.  Also MySQL is more widely known,
standardized, and simpler to learn for someone new to databases.

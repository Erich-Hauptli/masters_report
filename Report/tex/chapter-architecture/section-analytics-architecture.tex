\section{Career Path Architecture}
Once the databases are established and loaded, the real work can begin. 
ProGENitor takes in a request query and then using the career path modules,
graphs out the similar career paths taken by others who traversed through the
query point.  It also provides details about what was special about these
individuals, as they went through each vertex of their lives.  Then, through Weka,
the code draws out the complex events that had the largest impact towards the
the users passing through the query point.  As depicted in figure
\ref{fig:analytics_block}, the code is broken into four pieces.  One piece
gathers all of the vertices and defines the important edges between the
vertices.  Another piece of the code looks at the many different edges and attempts to
order the vertices in a manner that they can be graphed from left to right
without a lot of confusing edge crossings.  A third piece of code draws out the data
about each vertex and identifies the significant pieces of information that
separate the users in the query from everyone else who passed through the vertex. 
Finally, the last piece of code, uses Weka to draw out the complex action or
actions that had the greatest significance in causing the users to pass through
the query point.

\begin{figure}[H]
	\setlength{\unitlength}{0.1in} % selecting unit length
	\centering % used for centering Figure
	\begin{picture}(40,26) % picture environment with the size (dimensions)
		% 32 length units wide, and 15 units high.
		\put(-7,12.5) {User Query}
		\put(2.5,13){\vector(1,0){2}}
		\put(5,10.5){\framebox(12,5){Data Collection}}
		\put(17,12.5){\vector(1,-3){3}}
		\put(17,12.5){\vector(2,-1){4}}
		\put(17,12.5){\vector(2,1){4}}
		\put(17,12.5){\vector(1,3){3}}
		\put(21,21){\framebox(14,5){Edges}}
		\put(21,14){\framebox(14,5){Vertex Ordering}}
		\put(21,7){\framebox(14,5){Vertex Details}}
		\put(21,0){\framebox(14,5){Weka}}
		\put(35,23.5){\vector(2,-3){6}}
		\put(35,16.5){\vector(2,-1){4}}
		\put(35,9.5){\vector(2,1){4}}
		\put(35,2.5){\vector(2,3){6}}
		\put(37,12.5) {Results}
	\end{picture}
	\caption{Career Path Block Diagram} % title of the Figure
	\label{fig:analytics_block} % label to refer figure in text
\end{figure}
\subsection{Career Path Technology Stack}
As detailed in Chapter 1, the code is written in Java.  As the code for the
graphing can only show users paths taken and important pieces of data along the
way, Weka code was added to also derive insights based on combinations of data. 
Machine learning based on data can be very math intensive and complex.  There
are many different ways of looking at data sets.  To simplify this, the Weka
tool set was used in this project.  The tool set has many different algorithms
that can be easily implemented and applied.  In the case of ProGENitor, Weka has
only been applied to the education data; however, it could easily be
expanded to analyze additional data, such as the data about jobs.  Weka was
chosen as it has a well designed Java API and is open source.  One good alternative that could have been
used in place of Weka is RapidMiner.  Weka was chosen for the implementation as
there is significant documentation surrounding both tool sets and RapidMiner's
largest advantage, the graphical interface, is not applicable.

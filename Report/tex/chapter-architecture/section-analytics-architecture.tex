\section{Analytics Architecture}
Once the databases are established and loaded, the real work can begin. 
ProGENiotr takes in a JSON request query and then using the analytics modules,
maps out the similar career paths taken by others who traversed through the
query point.  It also provides details about what was special about these
individuals as they went through each node of their lives.  Then, through Weka,
the code draws out the events that had the largest impact towards the the users
that lead them to pass through the query point.  As depicted in figure
\ref{fig:analytics_block}, the analytics code is broken into four pieces.  One
piece gathers all of the nodes, determines the the transitional points, and
defines the important connections between the nodes.  Another piece of the code
looks at the many different transition points and attempts to order the nodes in
a manner that they can be mapped from left to right without a lot of confusing
crossings of interconnects.  A third piece of code draws out the data about each
node and identifies the significant pieces of information that seperate the
users in the query from everyone else who passed through the node.  Finally, the
last piece of code, uses Weka to draw out the action or actions that had the
greatest significance in the users passing through the query point.

\begin{figure}[H]
	\setlength{\unitlength}{0.1in} % selecting unit length
	\centering % used for centering Figure
	\begin{picture}(40,26) % picture environment with the size (dimensions)
		% 32 length units wide, and 15 units high.
		\put(-7,12.5) {JSON Query}
		\put(2.5,13){\vector(1,0){2}}
		\put(5,10.5){\framebox(12,5){Data Collection}}
		\put(17,12.5){\vector(1,-3){3}}
		\put(17,12.5){\vector(2,-1){4}}
		\put(17,12.5){\vector(2,1){4}}
		\put(17,12.5){\vector(1,3){3}}
		\put(21,21){\framebox(14,5){Node Ordering}}
		\put(21,14){\framebox(14,5){Node Details}}
		\put(21,7){\framebox(14,5){Node Connections}}
		\put(21,0){\framebox(14,5){Weka}}
		\put(35,23.5){\vector(2,-3){6}}
		\put(35,16.5){\vector(2,-1){4}}
		\put(35,9.5){\vector(2,1){4}}
		\put(35,2.5){\vector(2,3){6}}
		\put(37,12.5) {JSON Return}
	\end{picture}
	\caption{Analytics Block Diagram} % title of the Figure
	\label{fig:analytics_block} % label to refer figure in text
\end{figure}
\subsection{Analytics Technology Stack}
As detailed in the introduction to this chapter, the code is written in Java. 
As the code for the mapping can only show users paths taken and important pieces
of data along the way, Weka code was added to also derive insights based on
combinations of data.  Machine learning based on data can be very math intensive
and complex.  There are many different ways of looking at data sets.  To
simplify this, the Weka toolset was used in this project.  The toolset has many
different algorithms that can be easily implemented and applied.  In the case of
ProGENitor, Weka has only been applied to the education dataset; however, it
could easily be expanded to analyze additional datasets.  Weka was chosen as it
has a well designed Java API and it is open source.  One good alternative
that could have been used in place of Weka is RapidMiner.  In this case, as
there is significant documentation surrounding both toolsets and RapidMiner's
largest advantage, the graphical interface, is not applicable, Weka was chosen
for the implementation.

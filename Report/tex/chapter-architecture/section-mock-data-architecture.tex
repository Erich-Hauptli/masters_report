\section{Mock Data Architecture}
	To test and run ProGENitor, a database with data must be present.  As there was
not an existing one readily available, mock data needed to be generated and
populated into a database.  The SQL code allows for a file containing comma
separated values to be loaded into the database.  Thus, code to generate this
file with meaningful data was required.  
	
A Perl program was written to generate the .csv file that could be loaded into
the database.  Each line in the .csv file first indicates the database the line
should be loaded into and then the user id the data is associated with.  Each
individual user is assumed to have a distinct user id.  The subsequent data in
each row is built off a random selection from an array of data for each column.
For example, when generating the university for a particular user's education
node, a text file containing potential universities is loaded into an array and
then the value is randomly selected from the array.  This is done for each piece
of data loaded into the database, such that the database looks like it contains
real user data.  The randomness can be controlled and particular elements can be
weighted so they show up more frequently.  Additionally, the paths and
frequency users traverse through nodes can also be adjusted through variables
and the content within the text files.  This process will be stepped through in
detail in chapter 3.

\subsection{Mock Data Technology Stack}
Perl was the language chosen for the data generation for multiple reasons.  The
language excels at text manipulation.  The regular expressions are excellent, as
are the array processing capabilities.  As Perl is a scripting language, it is
not bound by the many rules surrounding Object Orientated language.  Thus,
there was no need to keep track of variable types and the code required to do
many things was much more concise.  The language also very easily manipulates
files.  For all of these reasons, creating the data set to be loaded into the
database could be done quickly and easily through Perl scripting.

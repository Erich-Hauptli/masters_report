\section{User Stories}
\label{sect:user-stories}
	As ProGENitor is a Java application that can be setup to parse through any SQL
database, it can easily be integrated into large scale social platform such as
LinkedIN or a smaller scale corporate database.  This provides it with two core
use cases.  The first use case focuses on guiding an individual through their
career based on the data from a large scale social platform.  The second use
case focuses on a corporate implementation that is largely used to ensure that talent is
prepared to step into future roles and responsibilities.
\subsection{Individual Career Planning}
		ProGENitor is an excellent tool to assist an individual in their own career
	planning.  If ProGENitor were implemented in a social networking framework such
	as LinkedIN, a vast amount of data could be provided to an individual to
	provide guidance towards meeting a specific goal.  For example, consider Tom
	the engineer wishes to some day be the lead engineer on his own project.  Using
	ProGENitor, he could pass a job title such as Chief Engineer to ProGENitor.  He
	would then be presented with a diagram graph showing how other users within
	the entire social network, not just Tom's social network, achieved this goal. 
	Additionally, the most common paths would be highlighted and the significant
	details about each job or education node would be displayed.  Finally,
	through Weka analytics, certain combinations of decisions would also be
	presented if they had a significant impact in achieving Tom's goal of
	becoming a Chief Engineer.  Tom could then use this information to make an
	educated decision about what actions might get him to his career goals fastest.
\subsection{Skill Identification}
		ProGENitor could also be used in a corporation to ensure that talent was
	always available to fill the jobs that were needed by the company.  The company
	could add performance review data into the database and then start identifying
	traits about successful employees that could be used to make hiring decisions. 
	Additionally, managers could use the feedback from ProGENitor to help guide
	their employees into gaining experience to aide them in moving into key roles. 
	
		For example, consider Carole a manager at a large technology firm.  She has
	to counsel employees in their careers and she needs to fill an already
	existing job vacancy.  Using ProGENitor she could identify what skills and
	traits make employees successful in her team and company to assist with her
	hiring decisions.  Also, when she is preparing to help guide employees career
	paths she can have concrete suggestions on actions they can take to achieve
	laid out career goals.  Both of these actions benefit Carole's company as it
	will lead to better trained and more satisfied employees.
\subsection{Financial Benefits}
One additional benefit companies would gain by	implementing ProGENitor would 	
be in an increase in desire among the users to maintain their own data.  Data
has become another form of currency in the digital age and there are many ways
companies profit from it.  If users see a direct advantage to adding to their
profiles and updating data about themselves they will likely do so.  This means
companies and social networks will increase the amount of data collected and
then be able to use it for other applications besides just using it for
ProGENitor.  For instance, LinkedIN collects a vast amount of data about its
users, but would likely benefit from having even more information, as the more
information that is available, the better they could target services and
advertisements towards their users.

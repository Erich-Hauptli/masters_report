\section{Sample Applications}
\label{sect:user-stories}
ProGENitor is an application that can be setup to parse through a vast amount
of data to provide insights into an individuals career.  It can easily be
integrated into large scale social platform such as LinkedIN or a smaller scale
corporate database.  This provides it with two core applications, which are
discussed in detail in Sections 1.3.1 and 1.3.2.

\subsection{Individual Career Planning}
ProGENitor is an excellent tool to assist an individual in their own career
planning.  Implemented in a social networking framework, ProGENitor can use the
vast amount of data in the site to provide individuals with
guidance towards meeting a specific goal.  

For example, consider Tom the engineer.  Tom wishes to some day be the lead
engineer on his own project.  Using ProGENitor, he could pass a job title such
as Chief Engineer into ProGENitor.  He would then be presented with a graph
showing how other users within the social network achieved this goal.  The most
common paths would be stand out in the graph and the significant details about
each job or education vertex could be displayed upon request.  Finally, through
Weka analytics, certain combinations of decisions would also be presented if
they had a significant impact in achieving Tom's goal of becoming a Chief
Engineer.  Tom could then use this information to make an  educated decision
about what actions might get him to his career goals fastest.

\subsection{Skill Identification}
A corporation could use ProGENitor to ensure that talent was always
available to fill the jobs that were needed by the company.  The company could
add performance review data into the database and then start identifying traits
about successful employees.  This information could then be used to make hiring
decisions.  Additionally, managers could use the feedback from ProGENitor to
help guide their employees into gaining experience to aid them in moving into
key roles.
	
For example, Carole, a manager at a large technology firm, has to counsel
employees in their careers and fill an already existing job vacancy.  ProGENitor would
improve her hiring decisions by identifying what skills and traits make
employees successful in her team and company.  When she is preparing to help
guide employees  on their career paths, ProGENitor would help her provide
concrete suggestions of actions the employee could take to achieve career goals.
The ProGENitor results  would benefit Carole's company by helping create better
trained and more satisfied employees.

\subsection{Financial Benefits}
ProGENitor would financially benefit companies by increasing the total data
they obtained from the user base.  Data has become another form of currency in
the digital age and there are many ways companies profit from it.  If users see
a direct advantage to adding to their profiles and updating data about
themselves they will likely do so.  This means companies and social networks
will increase the amount of data collected and then be able to use it for other
applications beyond ProGENitor.  For instance, LinkedIN collects a vast amount
of data about its users, but would likely benefit from having even more
information, as they could then better target their services and advertisements
towards their users.

\section{Conclusions}
\label{sect:conclusions}
If you are interested in doing a project similar to ProGENitor you should focus
on the data that you wish to present to the end user.  Focusing on relevant
insights is key as you want the end user to draw valid conclusions.  This is
difficult as you have draw from data about both the users who achieved the
career goal and those who did not.  As the group that failed to reach a goal is
vast, you must think about how to limit the data pulled to provide a result to
the user within a reasonable time frame.  In ProGENitor, the data set for the
users who failed to reach the goal vertex was limited to all the users who entered
a particular vertex of interest, thus conclusions could only be made about
activity within that vertex and about frequency of edges traveled by those users
who reached an end goal.  This leads to one area of improvement that could be
made to the ProGENitor results.  It would have been useful to also know about
the edges traveled for users who did not reach the goal, as it would allow the
end user to draw conclusions about career tracks that might be more successful
but less traversed.

Once you have the algorithms defined, the coding is relatively straight
basic.  As a novice programmer, I was able to complete the project in about 3
months with no incoming knowledge about databases or Weka and a basic knowledge
of Java programming.  Using synthetic data vastly sped up the whole process and
it gave me more control over testing my solution.  If you wish to go further on
this project, enhancing performance should be the main focus as that would allow
for more insights to be batched and could also provide for a better user
experience.  The core areas for these performance enhancements would come from
parallelization of the data analysis, optimization of the data fetches, and
running the workloads on better computing hardware.  A second key area of focus
should be on adding a user interface.  This would be required for the tool to
really be used by an end user, as without it, the data would be too cumbersome
for the user to consume.

All in all, ProGENitor delivered on the vision of providing users actionable
data built off of large career data sets.  Using the tool presents users with
insights into how others have achieved a particular goal and what some of the
key factors to achieving that goal were.  Using this information the user could
focus their efforts on the most important factors to achieving their goal and do
so in the most efficient manner possible.

\section{Future Work}
\label{sect:future-work}
Although ProGENitor is already to a point that it could quickly be deployed,
there are still a lot of improvements that could be made.  For
instance, creating a web user interface to visually demonstrate ProGENitor's
capabilities would go a long way to helping sell it to future customers. 
Another important work item would be to test and validate the software with
several different database types such as a NoSQL database.  Focusing on code
parallelization and on optimizing database pulls could greatly improve
performance.  Further improvement could be gained by pulling the node detailed
data only upon user request versus the current implementation, which pulls all
node data at once.

Currently, ProGENitor only looks at the education data for Weka.  Additional
insights could be gathered by generating more .arff files to be fed into
Weka.  This could be expanded to also look at the job data or other aspects of
the user data depending on the area the user was interested in.  To make this
feasible within a reasonable time window, Weka pulls would need to be attached
to an advanced insight request by the user.  Weka performance is far too slow to
have it be part of the initial career mapping query.

The quality of the results could be improved by growing the user data to include
information beyond education and employment.  The profiles could be grown to
include data about personality, work style, publications, or any other number of
useful pieces of information.  Finally, focusing on security and robustness by
adding in some testing would also be valuable for a deployable product.  Again,
ProGENitor could be deployed now with some minimal effort, but to deploy a
quality well performing product additional work should be implemented.

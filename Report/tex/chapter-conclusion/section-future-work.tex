\section{Future Work}
\label{sect:future-work}
Although ProGENitor is already to a point that it could be deployed relatively
quickly, there is still a lot of work that could be done in the future.  For
instance, creating a web UI to demonstrate ProGENitor's capabilities visually
would go a long way to helping sell it to future customers.  Another thing that
would be important would be to test and validate the software with several
different database types such as a NoSQL database.  Another area to focus on
would be performance.  There are some areas of the code that could be
parallelized and the database pulls could also be optimized.  Additionally, the
complete data pull on a node is currently pulling all nodes.  This could be
changed to only pull the data upon user requests.  As for the actual data
pulled, additional insight could be gathered by generating additional .arff
files to be fed into Weka.  To make this feasible in a reasonable time window,
Weka pulls would need to be attached to an advanced insight request by the user.
The performance is far too slow to have it be part of the initial query with the
career mapping. Currently, the code only looks at the education data for Weka;
this could be expanded to also look at the job data or other aspects of the user
data depending on the area the user was interested in.  Additionally, the user
data could be grown to include information beyond education and employment.  The
profiles could be grown to include data about personality, work style,
publications, or any other number of useful pieces of information.  Finally,
focusing on security and robustness by adding in some testing would also be
valuable for a deployable product.  Again, ProGENitor could likely be deployed
now with some minimal effort, but to deploy a quality well performing product
additional work would be needed.

\section{Related Work}
\label{sect:related-work}
A lot of career planning focuses around an individual forming a mind map or
taking quizzes to determine what they would like to do.  Next, individuals
are told to talk to people they know and look at existing jobs on job boards. 
Although these are valuable things that job seekers should do, it is not really
a proactive method to ensure that the individual develops the skills they need
to reach a desired job or career.  ProGENitor takes a different approach by
pulling all of the data in a career database, such as LinkedIn and then guides
users in how to reach these goals in the most efficient manner.  This is a much
more accurate method than asking a friend or neighbor or searching job boards. 
Several companies are starting to take similar approaches to ProGENitor, but at
the moment this is not a common method.

One example of a company doing something similar to ProGENitor is
TalentGuard\cite{talentguard}.  TalentGuard has a product that maps out career
paths in a similar fashion to ProGENitor.  First, the user feeds in a starting
and end point.  Then the tool generates the data in between these points.  This
requires the company to create the career paths and enter the data into the
tool.  It does not generate this data from existing users as ProGENitor does,
thus it requires some significant work on the company to deploy the tool.

Another example of a company doing something similar to ProGENitor is Mozilla. 
They wrote a program called Discover\cite{discover}, which uses their existing
OpenBadges\cite{openbadges} to generate career paths.  Currently this product is
just a prototype, but offers an alternate approach with a very polished and fun
user interface.  The tool expands upon what is looked at to include interests,
experiences, education, and personality traits.  It allows the user to view
other career paths, so that they can model a career path based on other
individual's careers.  Unlike ProGENitor, it does not provide an aggregate of
all of the users.  This means the user either has to do a lot of comparison
shopping, or they could end up mimicking someone who took an
inefficient of rarely traveled path to the end goal.  Additionally, it requires
the user to use the open badges, which limits the application to an online
community that uses the open badges tool.

A third example of a company that did something similar to ProGENitor was
LinkedIn.  They had a service called Career Explorer\cite{careerexplorer} which
allowed students to explore different career paths.  It gave the students the
ability to visualize career paths, identify people in their networks who could
assist with a career path, and provided other data from companies on the career
path.  One weakness of the tool is it did not draw from the whole breadth of the
LinkedIn database.  LinkedIn removed the tool when they went public as it was
not making enough money.  ProGENitor expands beyond students, draws from the
full database to provide all professionals with advice, and should be more
profitable due to a larger customer base.

All in all, though some tools do exist, there are not any that draw from the
vast amount of data collected today as ProGENitor does.  Additionally, though
some of these tools offer some similar features to ProGENitor, all have
significant disadvantages to what ProGENitor can offer.  ProGENitor offers a
scalable service that currently no other company has available.

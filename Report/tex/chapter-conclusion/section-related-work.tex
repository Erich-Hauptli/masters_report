\section{Related Work}
\label{sect:related-work}
A lot of career planning focuses around the individual forming a mind map or
taking quizzes to determine what they want to do.  Then talking to people they
know and possibly looking at job boards.  Although these are valuable things
that job seekers should do, it is not really a proactive method to ensuring that
the individual develops the skills they will need to reach a job or career path
they want to pursue.  ProGENitor instead takes the method of pulling all of the
data in a career database, such as LinkedIn and guides users how to reach these
goals in the most efficient manner.  This is a much more accurate method than
asking a friend or neighbor.  Several companies are starting to take similar
approaches to ProGENitor, but at the moment this is not a common method.

One example of a company doing something similar to ProGENitor is
TalentGuard\cite{talentguard}.  TalentGuard has a product that maps out career
paths in a similar fashion to ProGENitor.  The user feeds in a starting and end
point and the tool generates the data in between.  This requires the company to
create the career paths and enter the data into the tool though.  It does not
generate this data from existing users as ProGENitor does, thus it requires some
significant work on the company to deploy the tool.

Another example of a company doing something similar to ProGENitor is Mozilla. 
They wrote a program called Discover\cite{discover}, which uses their existing
OpenBadges\cite{openbadges} to generate career paths.  Currently this product is
just a prototype, but offers an alternate approach with a very polished and fun
user interface.  The tool expands upon what is looked at to include interests,
experiences, education, and personality traits.  It allows the user to view
other individual career paths, so that the user can model a career path based on
these other users.  Unlike ProGENitor, it does not provide an aggregate of all
of the users, so this either requires the user to do a lot of comparison shopping,
or the user could potentially end up mimicking someone who took an inefficient
of rarely traveled path to the goal the user is trying to reach.  Additionally,
it requires the user to use the open badges, which limits the application to an
online community that uses the open badges tool.

A third example of a company that did something similar to ProGENitor was
LinkedIn.  They had a service called Career Explorer\cite{careerexplorer} which
allowed students to explore different career paths.  It gave the students the
ability to visualize career paths, identify people in their networks who could
assist with a career path, and other data from companies on the career path. 
The tool did not draw from the whole breadth of the LinkedIn database and
LinkedIn removed the tool when they went public as it was not making enough
money.  ProGENitor expands beyond students, draws from the full database to
provide all professionals with advice and should be far more profitable as it
has a much larger customer base.

All in all, though some tools do exist, there are not any that draw from the
vast amount of data collected today as ProGENitor does.  Additionally, though
some of these tools offer some similar features to ProGENitor, all have
significant disadvantages to what ProGENitor can offer.  ProGENitor would offer
a service that currently no one offers that provides the ability to have it
deployed on both a smaller scale at a corporation or in a large scale social web
environment.

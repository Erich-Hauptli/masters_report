\section{Engineering Metrics}
\label{sect:engineering-metrics}
When looking at preparing to write the code for this project it is good to look
at about how much time will be required, how much code is needed to be written,
and what challenges will be faced.  This has been broken down by each major
piece of the code below.
\subsection{Databases}
Writing the code for creating and pulling from the databases took about 30
commits.  The code work took about 2 weeks or approximately 4\% of a man year. 
The most difficult part of this code was simply learning and using the SQL
database calls.  In total the code was approximately 1100 lines of code.
\subsection{Generating Synthetic Data}
Writing the code for generating the synthetic data took about 13 commits.  The
code work took about 1 weeks or approximately 2\% of a man year.  The most
difficult portion of this code was randomizing the data.  In total the code was
about 375 lines of code.
\subsection{Career Path Graph}
Writing the code for the career path graphing took about 11 commits.  The code
work took about 5 weeks or about 10\% of a man year.  This code has several
areas that were particular challenging.  One piece that was challenging, in the
vertex edges code, was pulling only the worst case vertex transitions from
the database.  Another challenging part of the code, in the vertex ordering
section, was ensuring that a vertex wasn't placed in a group if the prior vertex
wasn't already in a previous group.  Finally, in the vertex details code, pulling
the significant data from the total pieces of data was also challenging.  In
total the code was about 1050 lines of code.
\subsection{Weka Insights}
Writing the code for the Weka insights took only 3 commits.  The code work was
quick due to the ease of implementing the API.  It took less than 1 week to
implement or 1\% of a man year.  The code was not difficult to write as the
documentation gave clear examples on how to run Weka.  The most challenging part
was learning the Weka API and then choosing the analysis method.  In total the
code was about 150 lines of code.
\subsection{Total Code}
Combining all this code, there was approximately 60 commits and about 2700 lines
of code.  All this coding took about 17\% of a man year or about 2 solid months
of coding.  In reality, the project was worked on only part time and stretched
out to about three and a half months.  The most important decision in this
project was focusing on the graph analysis of the returned data and on ensuring
that the results and conclusions drawn from the results were valid. 
Extracting valid conclusions is extremely important, which is difficult to do
because you have to look at both the users who reached a goal and those who did
not.
\subsection{Version Control}
This project used GIT to manage the version control of the code.  This helped
greatly with managing the many aspects of the code, maintaining a change list,
and reverting code back to functional states when something went wrong.  If
ProGENitor became a multi-person project, GIT would become increasingly more
important as branching and merging would become very important.  Finally,
with a customer or multi-customer deployment, release trees would need to be
implemented to avoid releasing development code or customer directed code to all
customers.

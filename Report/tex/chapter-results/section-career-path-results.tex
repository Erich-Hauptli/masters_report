\section{Career Path Results}
\label{sect:career-path-results}
Examples of the JSON objects returned by the career mapping portion of the
ProGENitor code are shown below.  Each example does not contain a complete set
of data as that would be too much to show in this report.

\subsection{Node Interconnect JSON Results}
In the node interconnect JSON Object, an array of node interconnections is
returned.  Each element of the array contains a starting node and an ending node
for the transition.  Additionally, the array element also contains a transition
frequency.  The transition frequency indicates how often the transition occurs. 
Currently this is on a scale of 0-10, so that user data is not exposed to the
ProGENitor users.  Thus, the actual number of users transitioning from one node
to another node is divided by the total number of users that data was pulled for
the career map.  This percentage is then multiplied by 10 and rounded.

\noindent[\{"Node Connections":[\\*
\{"node A":"Bachelors","node B":"Masters","transition frequency":6\},\\*
\{"node A":"Masters","node 	B":"Circuit Designer","transition frequency":7\},\\*
\{"node A":"Circuit Designer","node B":"Block 	Owner","transition
frequency":8\},\\* \{"node A":"Block Owner","node B":"Design Owner","transition
frequency":9\},\\*
\ldots\\* 
\{"node A":"Coder","node B":"Function Lead","transition frequency":0\},\\*
\{"node A":"Function Lead","node B":"Masters","transition frequency":0\}]\},\\*


\subsection{Node Ordering JSON Results}
In the node ordering JSON Object, an array containing the order which the nodes
should be display is returned.  Each element of the array contains a node name
and the order number it should be displayed.  Thus, the nodes with an order of
1 should be the first nodes displayed in the career path map, then
moving sequentially up, each node in the group should be displayed until the
final node group is displayed.  This will allow the map to flow with minimum
interconnects flowing in the reverse order.

\noindent \{"Node Ordering":[\\*
	\indent\{"node	name":"Timing","order":"2"\},\\*
	\indent\{"node name":"Signal Integrity","order":"2"\},\\*
	\indent\{"node name":"Platform Chief Engineer","order":"7"\},\\*
	\indent\{"node name":"PHD","order":"5"\},\\*
	\indent\{"node name":"Entry Coder","order":"1"\},\\*
	\indent\ldots\\*
	\indent\{"node name":"Block Owner","order":"6"\},\\*
	\indent\{"node name":"Chiplet Designer","order":"1"\}]\},\\*


\subsection{Node Details JSON Results}
In the node detail JSON Object, an Array containing all of the various nodes
will be returned.  Each node will be nested object containing an array of data
points.  Examples of these data points are titles, companies, time spent at the
node, and any other points of interest within the database.  Each of these data
points will be an Object that also contains a nested JSON array.  This array
will then contain data about each data point, broken down into the percentage of
users who matched a specific piece of information for that data point.  For
example, shown below is a data point for the companies that users worked for
when they worked at a Timing job.  To protect user data, this is not shown as
number of users, but as the percentage of users who spent time working for one
company versus all users who spent time working at that particular job.  To
avoid having millions of entries, a threshold is set such that if the threshold
is not met, the data is lumped into an Other group.  This Other group would then
contain the total user data that did not meet the threshold.  Finally, a JSON
Object containing any significant data points is also returned.  ProGENitor
compares the users who reached the goal against all users who passed through the
node to determine what was statistically different from the users who reached
the target goal.  These differences are listed in the significant data object.

\noindent \{"Nodes Data":[\\*
	\indent \{"Node Name":"Timing","Node Data":[\\*
		\indent \{"Data Breakout":[\\*
		\indent \indent	\{"name":"Other","value":"0.9174312\%"\}\\*
		\indent	\indent \{"name":"Timing_all","value":"100.0\%"\}],\\*
		\indent	"Data Point Name":"title"\},\\*
		\indent\{"Data Breakout":[\\*
		\indent	\indent	\{"name":"Verizon","value":"100.0\%"\},\\*
		\indent	\indent	\{"name":"Verizon_all","value":"11.33721\%"\},\\*
		\indent	\indent	\{"name":"Cisco Systems_all","value":"12.790698\%"\},\\*
		\indent	\indent	\{"name":"Boeing_all","value":"7.5581393\%"\},\\*
		\indent	\indent	\{"name":"Hewlit-Packard_all","value":"8.139535\%"\},\\*
		\indent	\indent	\{"name":"IBM_all","value":"7.2674417\%"\},\\*
		
		\indent	\indent	\{"name":"General Motors_all","value":"8.72093\%"\},\\*
		\indent	\indent	\{"name":"General Electric_all","value":"7.2674417\%"\},\\*
		\indent	\indent	\{"name":"Microsoft_all","value":"8.72093\%"\},\\*
		\indent	\indent	\{"name":"Intel_all","value":"8.72093\%"\},\\*
		\indent	\indent	\{"name":"Lockheed Martin_all","value":"11.046512\%"\},\\*
		\indent	\indent	\{"name":"AT&T_all","value":"8.430233\%"\}],\\*
		\indent"Data Point Name":"company"\},\\*
		\indent\ldots\\*
		\indent \{"Significant":[]\}]\},\\*

\subsection{Example Carrer Maps}

\subsection{Career Path Performance}
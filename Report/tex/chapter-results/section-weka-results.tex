\section{Weka Results}
\label{sect:weka-results}
When running Weka, the results are not initially returned in a JSON Object. 
Weka returns the data in the following way shown below.  Then ProGENitor has to
parse the data and place the extracted insights into a JSON Object that can be
returned to the end user as part of the overall JSON Object.

\pagebreak 
\begin{tt}
\begin{footnotesize}
\noindent EM
\noindent ==\\*
\noindent Number of clusters selected by cross validation: 8\\*                           
\begin{tabular}{lcccccccc} 
&&&&&Cluster\\
Attribute&0&1&2&3&4&5&6&7\\
&(0.23)&(0.28)&(0.06)&(0)&(0.13)&(0.18)&(0.07)&(0.05)\\
\kern-1em degree\\
PhD&1.0&1.0029&1.04&1&1.0&1.0&1.03&478.9\\
Bachelors&1.0&2684&1.0&1&1.0015&1688&630&1.0\\
Masters&2164&1.0&589&1&1252&1.0&1.0&1.0\\
total&2166&2686&591&3&1254&1690&632&480.9\\
\kern-1em school\\
Duke&273&458&1.0&1&1.0&1.0&1.0&24\\
Stanford&1.0&1.0&164.3&1&163.7&231&239.9&48\\
USC&277&426&1.0&1&1.0&1.0&1.0&38\\
Berkeley&267&445&1.0&1&1.0&1.0&1.0&36\\
\ldots&&&&&&&&\\
Texas&1.0&1.0&1.3&1&202.7&442&1.05&23\\ 
MSU&270.2&231.9&1.2&1&186.5&203&1.1&53\\
MIT&254&155.7&219.4&1&144.6&157&154&77\\
CalTech&1.0&1.0&206.4&1&166.6&206.9&238.1&40\\ 
total&2175&2694&599&11&262&1698&640&489\\
\kern-1em specialization\\ 
Fusion&1.2&1&211&1&1.5&1.0&1.0&1.0\\ 
RF&1.0&1.0&1.0&1&1.0&1.0&1.0&20\\ 
Magnetics&1.0&1.0&1.0&1&1.0&1.0&1.0&22\\
Circuits&1.9&1.0&17.9&1&658&1.0&1.0&1.0\\
Analog&1.0&1.0&1.0&1&1.0&1.0&1.0&25\\ 
\ldots&&&&&&&&\\
Digital&1.0&1.0&1.0&1&1.0&1.0&1.0&26\\
total&2184.6&2704.4&609.2&21&1272&1708.2&650.4&498.9\\
\kern-1em goal\\ 
true&1.0&1.0&1.0&1&91&110&1.1&14\\
false&2164.6&2684.4&589&1&1162.2&1579.3&630.4&465.9\\ 
total&2165.6&2685.4&590.2&2&1253&1689.2&631.4&479.9\\
\end{tabular}
\\*=== Clustering stats for training data ===\\*
Clustered Instances\\*
\begin{tabular}{lrc} 
0&2163&( 23\%)\\*
1&2684&( 28\%)\\*
2&471&(  5\%)\\*
4&1369&( 14\%)\\*
5&1716&( 18\%)\\*
6&600&(  6\%)\\*
7&478&(  5\%)\\*
\end{tabular}
\\*
\noindent Log likelihood: -4.016
\end{footnotesize}
\end{tt}\\*

\vspace*{-13mm}
\subsection{Explanation of Weka Results}
In the example above, 212 user instances reached the end goal the query was
searching for.  This can be determined by adding up all the data in the goal
equals yes row and subtracting the number of columns.  Weka uses a minimum value of 1 for
each element in the columns, thus the total instances of a value within an arff
file would be the sum of the row minus the number of columns.  When it states that 212
instances reached the end goal, this does not mean there were 212 users who
reached the end goal.  The arff file treats each educational instances as a new
input, thus, when all is said in done for this particular arff file there are
9481 educational instances with the database.  This makes sense if you add up
all the rows for bachelor degrees, master degrees, and PhDs.  This math will
also result in 9481 instances, assuming you subtract one for each value.  One
other odd piece about the data is the fact that the numbers are not crisp
integers.  In the attempt to generate the various clusters, Weka assigns a
probability to each educational instance that it belongs in a cluster.  This
means the math will get complex and not always place a value perfectly in only
one cluster.  This causes the values to come close to integers, but some times
instances don't neatly fit within one cluster.

Once how the data is populated in the results and is understood; it can be used
to draw some educated conclusions from the various clusters.  As the interest
is in the users who reached the end goal is the focus of the work, any clusters that are equal to
1 for a goal of yes can immediately be ignored.  In this case, that leaves three
clusters.  Looking through cluster 4 for instances, shows a higher number of
users who obtained a Master's Degree in Circuits and attended Stanford, Texas,
MSU, MIT, or Caltech for this degree.  

Cluster 5 shows the same information, only instead the users obtained a
bachelor's degree.  The interesting thing of note between clusters 4 and 5 is
the slight drop in users who reached the goal who obtained the master's degree. 
The drop is not significant which implies getting the master's degree is still
very important for a user who wishes to reach the end goal.  One other thing to note,
in the above group of data, important information has been chopped as the total
master's degrees does not match the ones displayed.  This data was simply
shorted for the report, but would normally be displayed in the Weka results.  The same
is true for the other clusters.  

In the third cluster, cluster 7, only a few users show up as reaching the goal. 
This cluster shows students who obtained a PhD.  The school from which they
obtained the degree did not stand out in the cluster, but the degrees obtained
did.  The core degrees highlighted were RF, Magnetics, Analog, and Digital.  It
is worth noting however, that the results don't give us the ability to determine
which one of these degrees is important as all 4 instances have more users than
users who reached the goal while obtaining a PhD.  In any case, due to the
significant drop in users who obtained a PhD, it is clear that a PhD is helpful
but not required in reaching the goal node.

\subsection{Weka Performance}
For the Weka runs, the same 10 previous runs for the career path performance
testing were also used to estimate the Weka performance.  In figure
\ref{tab:weka-perf}, a couple things can be observed.  First, generating the
.arff file takes an insignificant time compared to the time it takes Weka to
analyze the data.  Second, the time it takes Weka to analyze the data takes far
too long to be part of batched request.  This would have to be an option that a
user specifically requests in addition to what ProGENitor typically runs.  Weka
would be one of the most likely pieces of ProGENitor to be sped up by running on
a server because it is strictly computational and not limited by database
accesses.  That being said, the average run currently takes about seven and a
half minutes, which would be far too long to ever be deployed to an end user. 
Thus, the server would have to significantly speed up the run over the
development laptop used to do this project to ever considering deploying Weka
within the ProGENitor tool.

 \begin{table}[H]
  \centering
  \begin{tabular}{|p{17mm}|p{16mm}|p{18mm}|p{19mm}|p{20mm}|}
  \hline
  \
  %heading
  Case&Matched Users&Arff\newline Generation&Weka&Total\\
  \hline\hline
  Platform Chief&109&159.3ms&280.5s&280.6s\\ \hline
  Civil\newline Degree&2684&155.7ms&739.0s&739.1s\\ \hline 
  Architect&2330&210.5ms&97.7s&97.9s\\ \hline
  Circuit Designer&675&154.2ms&441.1s&441.3s\\ \hline
  Worked For IBM&260&154.1ms&617.0s&617.1s\\ \hline
  Fission Degree&260&186.1ms&383.7s&383.9s\\ \hline
  Analog Degree&24&187.7ms&206.8s&207.0s\\ \hline
  Embedded&55&155.2ms&277.7s&277.8s\\ \hline
  Floor- \newline planning&49&158.0ms&251.6s&251.7s\\ \hline
  Circuit Designer&1401&280.0ms&1211.1s&1211.4s\\ \hline
  \hline\hline
  Minimum&24&154.1ms&97.7s&97.9s\\ \hline
  Maximum&2684&280.0ms&1211.1s&1211.4s\\ \hline
  Average&785&164.7ms&450.6s&450.7s\\ \hline
  \end{tabular}
  \caption{Weka Insight Generation Time}
  \label{tab:weka-perf}
\end{table}